
\chapter{Przegląd literatury}\label{ch:review}

% TP: TODO: Ten rozdział powinien być przeglądem literatury na tematy związane z pracą. Sugeruję utworzyć sekcje poświęcone następującym tematom:
% - Generowanie modeli PL z opisów zagadnień - alternatywne systemy działające na podobnej zasadzie -> OptiMUS
% - Odkrywanie modeli PL z innego rodzaju danych - metody/algorytmy syntezy/odkrywania/generowania modeli PL (i innych PM) na podstawie danych/przykładów/rozwiązań/informacji zwrotnej/uczenia aktywnego/itd.
% - Duże modele językowe dla generowania kodu - w szczególności modele dedykowane pod generowanie modeli PL, np. AMPL ma taką nakładkę na ChatGPT https://ampl.com/guide-to-using-chatgpt-for-ampl-models-and-streamlit-apps/ Warto referować jakiś benchmark, który pozwoli wskazać, który model generalnie lepiej się sprawdza
% W przeglądzie literatury powinniście porównać przeglądane prace z Waszą, w szczególności krótko napiszcie jakie są różnice, w czym Wasze podejście jest lepsze/inne. Im bardziej odległa tematycznie praca, tym to porównanie może być krótsze, albo nawet pominięte (np. przy odkrywaniu z przykładów rozwiązań)
% Obecna sekcja "Przeszukanie literatury" brzmi opis, w jaki sposób pozyskaliście wiedzę o PL/jak sami się uczyliście. To nie o to chodzi. Sugeruję usunąć; jeśli w ten sposób były pozyskiwane dane do bazy zagadnień optymalizacyjch (problemów), to raczej należy z tamtym opisem zintegrować tę informację.
% Obecna sekcja "Uczenie przez wzmacnianie" pasuje bardziej do opisu metody (Rozdział 3).
% Obecna sekcja "Podobne narzędzia - OptiMUS" może być zalążkiem dla Sekcji Generowanie modeli PL z opisów zagadnień, o której piszę wyżej. Trochę przerobiłem tekst, który tam jest. To musi być bardzo konkretnie napisane, bez nieistotnych informacji i lania wody.




\section{Przeszukanie literatury}

Literaturą, którą wykorzystano do przygotowywania modeli \textit{PL} są powszechnie dostępne artykuły edukacyjne. Owe artykuły edukacyjne wyjaśniają zagadnienie \textit{PL} korzystając z przystępnych oraz łatwych w analizie przykładów. Artykuły etapowo wyjaśniają tworzenie modelu \textit{PL}, omawiając go parami: \textit{tekst problemu -- model}. Taka forma jest szczególnie korzystna do nauki \textit{DJM}. Poniżej przedstawiono przykład z jednego z artykułów edukacyjnych\cite{cimt}:

\subsubsection*{Opis problemu:} \label{sec:model_example}
\begin{quote}
Mała firma zajmuje się budowaniem altan ogrodowych.

Typ A wymaga nakładu 2 godzin pracy maszyny oraz 5 godzin pracy robotnika.  

Typ B wymaga nakładu 3 godzin pracy maszyny oraz 5 godzin pracy robotnika. 

Każdego dnia firma dysponuje łącznie 30 godzinami pracy maszyn oraz 60 godzinami pracy robotników. Zysk z każdej altany typu A to 60 funtów, a z każdej altany B to 80 funtów. Celem jest maksymalizacja profitu firmy.

\end{quote}

\subsubsection*{Zmienne decyzyjne}

$x$ – liczba altan typu A wyprodukowanych każdego dnia,

$y$ – liczba altan typu B wyprodukowanych każdego dnia.

\subsubsection*{Ograniczenia}

Czas pracy maszyn: $2x + 3y \leq 30$

Czas pracy robotników: $5x + 5y \leq 60$

oraz $x \geq 0, \ y \geq 0$


\subsubsection*{Funkcja celu}
\[
    P = 60x + 80y
\]

\subsubsection*{Gotowy model problemu \textit{PL}}
    Zmaksymalizuj $P = 60x + 80y$

przy ograniczeniach:

    $2x + 3y \leq 30$
    
    $x + y \leq 12$
    
    $x  \geq 0$
    
    $y \geq 0$

forma wyroczni (uczenie ze wzmocnieniem); poszukać prace

\section{Uczenie przez wzmacnianie}
Uczenie przez wzmacnianie jest jednym głównych typów uczenia maszynowego, który polega na interakcji ze środowiskiem. Uczenie przez wzmacnianie jest problemem, z którym musi zmierzyć się agent. Zadaniem agenta jest nauka zachowania poprzez interakcje z otoczeniem, metodą prób i błędów.\cite{kaelbling1996reinforcement}

W pracy wykorzystano uczenie przez wzmacnianie, którego kroki opisano poniżej:
\begin{itemize}
    \item \textit{DMJ} generuje model \textit{PL} w języku \textit{ZIMPL} na podstawie opisu słownego opisu problemu \textit{PL},
    \item Niezależny \textit{DMJ} ocenia jakość wygenerowanego modelu \textit{PL} na podstawie ścisłych wytycznych oraz poprawnej wersji kodu sporządzonej przez programistów
    \item Na podstawie oceny wygenerowanego modelu \textit{PL}, instrukcja generacji modelu jest poprawiana
\end{itemize}

\section{Podobne narzędzia -- \textit{OptiMUS}} \label{sec:optimus}

\akronim{OptiMUS} \cite{ahmaditeshnizi2023optimus} jest systemem opartym o \akronim{DMJ}, który generuje modele PL w języku TODO na podstawie opisu słownego zagadnienia w języku angielskim. % TP: TODO <- uzupełnijcie język/format, w którym działa OptiMUS
Głównymi różnicami między narzędziami jest obsługa modeli \textit{ang. mixed-integer linear programming (MILP)} oraz system automatycznej iteracji w~celu naprawy błędów (\akronim{OptiMUS}). % TP: TODO <- nie jest jasne, który system obsługuje MILP, a który nie; wydawało mi się, że Wasz obsługuje MILP; z punktu widzenia generowania kodu obsługa MILP nie jest szczególnie trudna.
OptiMUS w~razie uzyskania niepoprawnej odpowiedzi wysyła zapytanie do \akronim{DMJ} dodatkowo z~treścią otrzymanego błędu. % TP: TODO <- nie jest jasne skąd się bierze błąd
Program generuje wtedy poprawiony kod --- operacja jest wykonywana do momentu uzyskania poprawnej odpowiedzi, bądź do momentu osiągnięcia maksymalnej liczby iteracji. Dodatkowo \akronim{OptiMUS} zapisuje dane z tekstu do pliku w określonym formacie, co ułatwia przetwarzanie. % TP: TODO <- jakie dane? Czy to jest istotna informacja?



