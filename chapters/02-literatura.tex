
\chapter{Podstawy teoretyczne}

\section{Struktura Pracy}

Struktura pracy jest następująca. W rozdziale 2 przedstawiono przegląd literatury na temat programowania liniowego oraz podanie przykładu rozwiązania. Rozdział 3 jest poświęcony opisowi naszego rozwiązania oraz poglądowym schematem struktury rozwiązania. Rozdział 4 zawiera opis zbioru danych. Tłumaczy on strukturę pracy, tworzenie odpowiedzi w formacie kodu \textit{ZIMPL} oraz opisuje przebieg rozwoju oraz przetwarzania zbioru danych. W rozdziale 5 przedstawiono pytania eksperymentalne oraz odpowiedzi na nie, wraz ze statystykami przebiegu eksperymentów. Rozdział 6 omawia zalety oraz wady rozwiązania, proces poprawy rozwiązania jak i napotkane trudności związane z przebiegiem pracy. W rozdziale 7 znajduje się podsumowanie zawierające wnioski oraz dalszy kierunek prac.

\section{Przeszukanie literatury}

Literaturą, którą wykorzystano do przygotowywania modeli \textit{PL} są powszechnie dostępne artykuły edukacyjne. Owe artykuły edukacyjne wyjaśniają zagadnienie \textit{PL} korzystając z przystępnych oraz łatwych w analizie przykładów. Artykuły etapowo wyjaśniają tworzenie modelu \textit{PL}, omawiając go parami: \textit{tekst problemu -- model}. Taka forma jest szczególnie korzystna do nauki \textit{DJM}. Poniżej przedstawiono przykład z jednego z artykułów edukacyjnych\cite{cimt}:

\subsubsection*{Opis problemu:} \label{sec:model_example}
\begin{quote}
Mała firma zajmuje się budowaniem altan ogrodowych.

Typ A wymaga nakładu 2 godzin pracy maszyny oraz 5 godzin pracy robotnika.  

Typ B wymaga nakładu 3 godzin pracy maszyny oraz 5 godzin pracy robotnika. 

Każdego dnia firma dysponuje łącznie 30 godzinami pracy maszyn oraz 60 godzinami pracy robotników. Zysk z każdej altany typu A to 60 funtów, a z każdej altany B to 80 funtów. Celem jest maksymalizacja profitu firmy.

\end{quote}

\subsubsection*{Zmienne decyzyjne}

$x$ – liczba altan typu A wyprodukowanych każdego dnia,

$y$ – liczba altan typu B wyprodukowanych każdego dnia.

\subsubsection*{Ograniczenia}

Czas pracy maszyn: $2x + 3y \leq 30$

Czas pracy robotników: $5x + 5y \leq 60$

oraz $x \geq 0, \ y \geq 0$


\subsubsection*{Funkcja celu}
\[
    P = 60x + 80y
\]

\subsubsection*{Gotowy model problemu \textit{PL}}
    Zmaksymalizuj $P = 60x + 80y$

przy ograniczeniach:

    $2x + 3y \leq 30$
    
    $x + y \leq 12$
    
    $x  \geq 0$
    
    $y \geq 0$

forma wyroczni (uczenie ze wzmocnieniem); poszukać prace

\section{Uczenie przez wzmacnianie}
Uczenie przez wzmacnianie jest jednym głównych typów uczenia maszynowego, który polega na interakcji ze środowiskiem. Uczenie przez wzmacnianie jest problemem, z którym musi zmierzyć się agent. Zadaniem agenta jest nauka zachowania poprzez interakcje z otoczeniem, metodą prób i błędów.\cite{kaelbling1996reinforcement}

W pracy wykorzystano uczenie przez wzmacnianie, którego kroki opisano poniżej:
\begin{itemize}
    \item \textit{DMJ} generuje model \textit{PL} w języku \textit{ZIMPL} na podstawie opisu słownego opisu problemu \textit{PL},
    \item Niezależny \textit{DMJ} ocenia jakość wygenerowanego modelu \textit{PL} na podstawie ścisłych wytycznych oraz poprawnej wersji kodu sporządzonej przez programistów
    \item Na podstawie oceny wygenerowanego modelu \textit{PL}, instrukcja generacji modelu jest poprawiana
\end{itemize}

\section{Podobne narzędzia -- \textit{OptiMUS}\cite{ahmaditeshnizi2023optimus}}

Podczas przeglądu literatury natrafiono na podobne narzędzia wykonane przez Wydział Nauk o Zarządzaniu i Inżynierii na Uniwersytecie Stanforda -- \textit{OptiMUS}. Głównymi różnicami między narzędziami jest możliwość obsługi problemów \textit{ang. mixed-integer linear programming (MILP)} oraz system automatycznej iteracji w celu naprawy błędów (\textit{OptiMUS}). Program w razie uzyskania niepoprawnej odpowiedzi wysyła zapytanie do \textit{DMJ} dodatkowo z treścią otrzymanego błedu. Program generuje wtedy poprawiony kod --- operacja jest wykonywana do momentu uzyskania poprawnej odpowiedzi bądź do momentu osiągnięcia maksymalnej liczby iteracji. Dodatkowo \textit{OptiMUS} zapisuje dane z tekstu do pliku w określonym formacie, co ułatwia przetwarzanie. \label{sec:optimus}


