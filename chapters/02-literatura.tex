
\chapter{Podstawy teoretyczne}

Jak tworzymy strukturę i dlaczego

\section{Przeszukanie literatury}
Przeszukanie literatury na temat tworzenia modeli programowania liniowego; przykłądy rozwiązań, forma wyroczni (uczenie ze wzmocnieniem); poszukać prace

Rozdział teoretyczny --- przegląd literatury naświetlający stan wiedzy na dany temat. 

Przegląd literatury naświetlający stan wiedzy na dany temat obejmuje rozdziały pisane na podstawie
literatury, której wykaz zamieszczany jest w części pracy pt.~\emph{Literatura} (lub inaczej \emph{Bibliografia}, \emph{Piśmiennictwo}).

W tekście pracy muszą wystąpić odwołania do wszystkich pozycji zamieszczonych w
wykazie literatury. \textbf{Nie należy odnośników do literatury umieszczać w stopce strony.} Student jest
bezwzględnie zobowiązany do wskazywania źródeł pochodzenia informacji przedstawianych w pracy,
dotyczy to również rysunków, tabel, fragmentów kodu źródłowego programów itd. Należy także podać
adresy stron internetowych w przypadku źródeł pochodzących z Internetu.


\section{Budowanie modeli z}
\subsection{Tekstów}
OptiMUS porównanie, odniesienie (różnice, w czym jesteśmy lepsi/różni)

\subsection{Przykładów}

\subsection{Innych źródeł}


