
\chapter{Zbiór danych}\label{ch:dataset}

Na potrzeby metod opisanych w Rozdziale~\ref{ch:generation}. przygotowano zbiór danych składający się z~zagadnień optymalizacyjnych przedstawionych w języku angielskim oraz referencyjnych modeli PL w języku \akronim {ZIMPL} dla tych zagadnień. %Zbiór danych stanowią zadania z treścią z dziedziny PL wraz z ich wynikiem zapisanym w języku \akronim {ZIMPL}. % TP: <- powtórzenie

Zbiór danych zawiera 4380 przykładów wraz z prawidłowym kodem \akronim {ZIMPL}. Stanowi on zwielokrotnienie 40 unikalnych przykładów bazowych, stworzonych poprzez przeszukiwanie dostępnych źródeł dotyczących problemów programowania liniowego \cite{brilliant_linear,byjus_linear,cimt,arsdcollege2020,libretexts_linear,superprof_linear,toppr_graphical}. Przykłady bazowe posiadają dwie wersje dostępnych rozwiązań: sztywne oraz parametryzowane, zatem dla 40 unikalnych treści w~rzeczywistości powstaje zbiór 80 przykładów. Dla każdego unikalnego przykładu zmieniono jego treść pod względem wartości parametrów i zbiorów, funkcji celu oraz opisu zadania. Zachowano podobną logikę zadania. Stworzono od 8 do 12 różnych opisów zadania, w~zależności od liczby danych zawartych w~bazowym zadaniu. Dla każdego z~tak utworzonych zadań oraz zadania bazowego, na podstawie którego zostały utworzone, zmieniono strukturę opisu problemu na pięć różnych sposobów, zapewniając możliwość testów takich samych zadań zależnie od wariantu opisu tekstowego problemu. Podane widełki wynikają z usuwania przykładów, które w~swojej końcowej formie nie przedstawiały poprawnego opisu zadania, zwłaszcza pod względem zbyt małej ilości podanych danych, względem kodu, któremu odpowiadały.
% TP: wypada podać dokładniejsze statystyki, tj. bazowych zagadnień jest mniej (podajcie ile) i dla każdego stworzono od X do Y wariantów opisu oraz wzorcowych modeli. Łacznie przykładów jest 4380. - DONE

Zbiór danych podzielono i wykorzystano do odpowiednio: trenowania, walidacji oraz testowania programu. Zbiór treningowy wyselekcjonowano jako 20 przykładów pochodzących od 10 różnych zadań bazowych, wybranych tak, aby pokryć jak największy obszar zróżnicowanych elementów kodu \akronim {ZIMPL}. Zbiór walidacyjny składa się z pozostałych bazowych przykładów. Łącznie zawiera w sobie 60 przykładów w tym 30 przykładów z odpowiedzią w formie sztywnego kodu oraz 30 przykładów kodu parametryzowanego. Tak jak w przypadku doboru zbioru treningowego, przyjęto zasadę wyboru zadań pochodzących od zróżnicowanych zadań bazowych. Zbiór testowy stanowi 4300 przykładów zwielokrotnionych przykładów wraz z kodem \akronim {ZIMPL} w zróżnicowanej formule.  % TP: wg jakiego schematu podzielony? - DONE

% TP: unikajcie synonimów. Skoro zbiór danych został wprowadzony jako "zbiór danych" to niech zawsze nazywa się zbiorem danych, a nie bazą. Poprawiam, ale być może coś zostanie. - mam nadzieję, że DONE
\section{Proces powstawania zbioru danych}

\begin{figure}
\centering
\begin{tikzpicture}[
    node distance=1.5cm and 2cm,
    every node/.style={align=center},
    stage/.style={rectangle, draw=blue!70, fill=blue!10, thick, minimum height=1cm, minimum width=4.5cm, rounded corners},
    stagevol2/.style={rectangle, draw=green!70, fill=green!10, thick, minimum height=1cm, minimum width=5cm, rounded corners},
    ending/.style={rectangle, draw=lightgray!70, fill=lightgray!10, thick, minimum height=1cm, minimum width=5cm, rounded corners},
    dataset/.style={rectangle, draw=red!70, fill=red!10, thick, minimum height=1cm, minimum width=5cm, rounded corners},
    frame/.style={draw=black!70, thick, rounded corners, inner sep=1cm},]
    arrow/.style={->, thick}
]

\node[stage] (search) {\textbf{Etap wyszukiwania} \\ zagadnień optymalizacyjnych};
\node[stage, below=of search] (model) {\textbf{Etap formułowania} \\ modeli PL \\ (w \textit{ZIMPL})};

\node[stagevol2, below=of model, minimum height=0.5cm, text depth=0.5ex] (multiplication) {\textbf{Etap zwielokrotniania przykładów}};

\node[frame, below=of model, minimum width=8cm, minimum height=6cm, anchor=north] (frame) {};

\node[stagevol2, below=1cm of frame.north] (descVariants) {\textbf{Tworzenie wariantów} \\ \textbf{opisu tekstowego} \\ (12 opisów)};
\node[stagevol2, below=of descVariants] (modelVariants) {\textbf{Tworzenie wariantów} \\ \textbf{struktury modelu} \\ (5 modeli)};

\node[dataset, below=of frame.south] (check) {\textbf{Sprawdzenie poprawności} \\ \textbf{treści zadań i kodu} \\ przez ekspertów \\ oraz przy użyciu SCIP \\ \textbf{Usuwanie błędnych przykładów}};
\node[dataset, below=of check] (dataset) {\textbf{Dodanie do zbioru danych} \\ od 50 do 60 przykładów};

\node[ending, below=of dataset] (end) {\textbf{Gotowy zbiór} \\ 4380 przykładów};

\draw[arrow] (search) -- (model);
\draw[arrow] (model) -- (multiplication.north);
\draw[arrow] (multiplication.south) -- (descVariants);
\draw[arrow] (descVariants) -- (modelVariants);
\draw[arrow] (modelVariants) -- (check);
\draw[arrow] (check) -- (dataset);
\draw[arrow] (dataset) -- (end);

\draw[arrow] (dataset.east) -- ++(3cm, 0) -- ++(0, 18.25cm) -- ++(-5.5cm, 0) -- ++(0, -1cm);

\end{tikzpicture}
\caption{Proces tworzenia zbioru danych z podziałem na etapy.}
\label{fig:dataset-creation-logical}
\end{figure}

Proces tworzenia zbioru danych podzielono na trzy etapy zwizualizowane na Rysunku~\ref{fig:dataset-creation-logical} i~opisane w~kolejnych sekcjach: etap wyszukiwania zagadnień optymalizacyjnych, etap formułowania modeli PL przy użyciu języka \textit{ZIMPL}, oraz etap zwielokrotniania przykładów za pomocą zmiany sposobu formułowania opisów i~modeli PL. % TP: <- niejasne: "treści zadań". Czy chodzi o "...formułowania opisów i modeli PL."? - tak, DONE

% TP: być może warto rozrysować ten proces razem z wariantami/równoległymi ścieżkami? - DONE, nie ma równoległych ścieżek


\subsection{Wyszukiwanie zagadnień optymalizacyjnych}

% TP: tutaj należy opisać protokół wyszukiwania; obecnie nie wiadomo skąd się wzięły akurat takie źródła - dlaczego je wybraliście. Wybór źródeł powinien być przeprowadzony w jakiś systematyczny sposób, np. na podstawie konkretnych zapytań wrzuconych w wyszukiwarkę google scholar/scopus/web of science/google books i odfiltrowaniu wyników w usystematyzowany sposób. - Adam x2?

Ze względu na specyfikę zbioru danych, skupiono się na przeszukaniu źródeł opisów zagadnień optymalizacyjnych zawierających sformułowania wzorcowych modeli PL.

Głównym źródłem zagadnień optymalizacyjnych i modeli PL jest dokumentacja języka  \akronim{ZIMPL}~\cite{zimpl_manual}, która została napisana w formie przypominającej instrukcję. Literaturą, którą wykorzystano do przygotowywania modeli \textit{PL} są powszechnie dostępne artykuły edukacyjne \cite{brilliant_linear,byjus_linear,cimt,arsdcollege2020,libretexts_linear,superprof_linear,toppr_graphical}. Owe artykuły wyjaśniają zagadnienie \textit{PL} korzystając z przystępnych oraz łatwych w analizie przykładów. Artykuły etapowo wyjaśniają tworzenie modelu \textit{PL}, omawiając go parami: \textit{tekst problemu -- model}. Zadania są zapisane w języku angielskim. Ich formuła jest zróżnicowana. Zadania są przedstawiane w formie długich opisów, punktowania problemu, tabelek, bądź są połączeniem tych form. Łącznie stworzono 40 unikalnych przykładów. 

Poniżej przedstawiono przykład wzorcowego modelu \textit{PL} wraz z opisem zagadnienia optymalizacyjnego planowania produkcji~\cite{cimt}:

% TP: TODO: Tutaj jest kilka problemów:
% 1. Wcześniej piszecie, że opisy problemów są po angielsku, a tutaj jest przykład polski; sugeruję raczej użyć angielski ze zbioru
% 2. Sekcja "Gotowy model problemu PL" jest powtórzeniem trzech wcześniejszych sekcji.
% Sugeruję zostawić tylko sekcję "Gotowy model problemu PL" i wprowadzić do niej komentarze z wcześniejszych sekcji o elementach tego modelu.
\subsubsection*{Opis zagadnienia optymalizacyjnego:} \label{sec:model_example}
\begin{quote}
Mała firma zajmuje się budowaniem altan ogrodowych.

Typ A wymaga nakładu 2 godzin pracy maszyny oraz 5 godzin pracy robotnika.  

Typ B wymaga nakładu 3 godzin pracy maszyny oraz 5 godzin pracy robotnika. 

Każdego dnia firma dysponuje łącznie 30 godzinami pracy maszyn oraz 60 godzinami pracy robotników. Zysk z~każdej altany typu A to 60 funtów, a~z~każdej altany B to 80 funtów. Celem jest maksymalizacja zysku firmy.
\end{quote}

\subsubsection*{Zmienne decyzyjne}

$x$ – liczba altan typu A wyprodukowanych każdego dnia,

$y$ – liczba altan typu B wyprodukowanych każdego dnia.

\subsubsection*{Ograniczenia}

Czas pracy maszyn: $2x + 3y \leq 30$

Czas pracy robotników: $5x + 5y \leq 60$

oraz $x \geq 0, \ y \geq 0$


\subsubsection*{Funkcja celu}
\[
    P = 60x + 80y
\]

\subsubsection*{Gotowy model problemu \textit{PL}}
    Zmaksymalizuj $P = 60x + 80y$

przy ograniczeniach:

    $2x + 3y \leq 30$
    
    $x + y \leq 12$
    
    $x  \geq 0$
    
    $y \geq 0$


\subsection{Tworzenie modeli PL w języku \akronim{ZIMPL}}

Dla wyszukanych i sprawdzonych modeli PL stworzono ich reprezentacje w języku \akronim{ZIMPL} opisujące zagadnienie. %, rozwiązujące przedstawiony problem. % TP: model nic nie rozwiązuje, model opisuje zagadnienie - DONE
Każdy model PL posiada dwa różne warianty w języku  \akronim{ZIMPL}: wariant w~formule programowania sztywnego oraz w~formule parametryzowanej. Dokładny opis specyfiki danego wariantu został przedstawiony w rozdziale~\ref{ch:generation:generating}

Obie wersje kodu poprawnie rozwiązują problemy programowania liniowego. W czasie trenowania, walidacji i testowania generatora wykorzystywane są dane uwzględniające oba rozwiązania. Przy generowaniu modelu \akronim{ZIMPL} użytkownik wybiera formułę, a następnie, zależnie od jego wyboru, proces generacji przechodzi przez ciąg zapytań do DMJ, w wyniku których powstają opisane powyżej elementy kodu. % TP: TODO <- czy generator coś testuje? Tutaj jest chyba pomieszane nazewnictwo. Czy choci o proces tworzenia zbioru danych? Słowo "generować" pojawia się w kontekście problemu generowania modelu PL i metody generownia modelu PL, więc należy go unikać w innych kontekstach - DONE

Przygotowany przykład, zanim trafi do kolejnego etapu poszerzania zbioru danych, jest sprawdzany za pomocą narzędzi do automatycznego sprawdzania jakości kodu oraz ręcznej analizy jakości kodu. Poprawność składniowa oraz semantyczna modelu PL jest weryfikowana poprzez próbę jego rozwiązania solverem SCIP \cite{scip_documentation}. Model PL, dla którego SCIP zwrócił rozwiązanie dopuszczalne, jest uznawany za poprawny składniowo i semantycznie. Weryfikacja przebiegała również poprzez ręczne sprawdzanie przez dwóch niezależnych od siebie ekspertów każdego modelu i weryfikacji, czy poszczególne elementy kodu odpowiadają celowi optymalizacji, zależnościom między modelowanymi wielkościami oraz czynnikom podlegającym optymalizacji. Ponadto eksperci weryfikowali rozwiązanie optymalne modelu PL pod względem zgodności z wiedzą dziedzinową.

\subsection{Zwielokrotnianie przygotowanych przykładów}

Posiadając przygotowane 40 unikalnych przykładów wraz z ich modelami PL w formule sztywnego i parametryzowanego programowania, należało powiększyć zbiór o przykłady podobne, lecz różniące się sformułowaniem i wynikami.

Początkowy proces polegał na tworzeniu przykładów o zbliżonej strukturze, ale z dywersyfikowaną treścią fabularną oraz danymi. Na podstawie jednego przykładu tworzone było do 12 różnych przykładów, w których przekształcana była narracja zadania, mógł zmienić się cel zadania z maksymalizacji na minimalizację i odwrotnie, a także zmieniały się dane i ich ilość. Dzięki temu, z podobnego zadania powstawały zupełnie niezależne i nowe problemy programowania liniowego. Sumarycznie utworzono 800 unikalnych zadań wraz z ich zmodyfikowanym kodem  \textit{ZIMPL}.
Dokładny proces tworzenia nowych wariantów został przedstawiony poniżej:

\begin{enumerate}
\item Wybór bazowego przykładu do zwielokrotniania --- każde zadanie opiera się na poprawnie stworzonym i zweryfikowanym przykładzie ze zbioru 40 głównych przykładów. Przykłady posiadają zapisany i zweryfikowany kod w \akronim{ZIMPL}.
\item Losowa modyfikacja poszczególnych elementów zadania i kodu bazowego --- na podstawie przykładu bazowego można zastosować poniższe modyfikacje, które prowadzą do nowych, unikalnych zadań.
\begin{enumerate}
\item Zmiana narracji fabularnej --- zmiana kontekstu problemu (np. fabryka mebli na wartości odżywcze w posiłku) oraz wprowadzenie nowych nazw zmiennych decyzyjnych i zasobów.
\item Modyfikacja danych liczbowych --- zmiana wartości parametrów w ograniczeniach oraz zmiana współczynników celu.
\item Zmiana struktury ograniczeń --- dodanie nowych ograniczeń (nowe limity lub zasoby) oraz usunięcie istniejących ograniczeń.
\item Zmiana celu optymalizacji --- przekształcenie z maksymalizacji na minimalizację lub odwrotnie. Stosowane losowo dla połowy generowanych przykładów.
\item Dodanie nowych zmiennych decyzyjnych --- wprowadzenie nowego produktu/usług/zmiennej do treści zadania oraz kodu wynikowego.
\end{enumerate}
\item Weryfikacja poprawności zadania - zadanie musi posiadać pełny opis, na podstawie którego można stworzyć docelowy model w \akronim{ZIMPL}. Rozwiązanie w formie kodu powinno być matematycznie i składniowo poprawne.
\item Proces zostaje powtórzony dla kolejnych zadań bazowych.
\end{enumerate}

Posiadając taki zbiór danych, kolejnym etapem było stworzenie różniących się opisów. Każdy przykład był przepisywany i formułowany ponownie na jeden z pięciu sposobów. Dokładny opis elementów etapu znajduje się poniżej.

\begin{enumerate}
\item Wybór przykładu do zwielokrotniania --- uwzględniając przy tym wszystkie przykłady, które przeszły etap tworzenia różnych wariantów opisu tekstowego wraz z modyfikacjami kodu oraz przykłady bazowe.
\item Zwielokrotnianie wybranego przykładu na każdy z przedstawionych poniżej sposobów. Modyfikacji ulega jedynie treść zadania PL, kod \akronim{ZIMPL} będący rozwiązaniem jest przepisywany z wybranego przykładu w nienaruszonej formie.
\begin{enumerate}
\item Opis w formie punktów --- wersja przedstawiająca kluczowe informacje w podpunktach. Każdy element zadania jest krótko i konkretnie opisany. Zadanie prezentuje się czytelnie, a rozwiązujący ma jasno wyznaczone cele zadania.

Kryteria tworzenia:
	\begin{itemize}
		\item Wszystkie dane liczbowe i techniczne są przedstawiane w punktach.
		\item Każdy punkt opisuje tylko i wyłącznie jeden element zadania, w tym zasoby, ograniczenia, cel optymalizacji.
		\item Średnia długość każdego punktu wynosi 4 słowa.
		\item Wykorzystywane są rzeczowniki, liczby i jednostki miar, natomiast nie wykorzystuje się czasowników.
	\end{itemize}
	
\item Długi opis tekstowy --- zadanie wprowadza wartość fabularną oraz opisuje dokładnie zadanie w formie tekstowej. Dane techniczne są wplecione w narracyjny opis zadania.

Kryteria tworzenia:
	\begin{itemize}
		\item Opis zawiera szczegółową narrację i wprowadzenie kontekstu fabularnego.
		\item Wszystkie dane potrzebne do wykonania zadania oraz funkcja celu są wplecione w~tekst.
		\item Średnia długość zdań wynosi 10-16 słów.
		\item W opisie wykorzystuje się czasowniki, przymiotniki i rzeczowniki w celu dokładnego opisu fabularnego.
	\end{itemize}

\item Zwięzły opis tekstowy --- opis zwięźle przedstawiający problem w formie tekstowej. Nie zawiera opisów fabularnych, wprowadza jedynie konkretne informacje potrzebne do rozwiązania zadania.

Kryteria tworzenia:
	\begin{itemize}
		\item Opis przedstawia problem w skondensowanej formie ciągłej, ograniczając narrację fabularną do nazw poszczególnych zasobów.
		\item Wszystkie dane potrzebne do wykonania zadania oraz funkcja celu są wplecione w~tekst.
		\item Średnia długość zdania wynosi 8-12 słów.
		\item W opisie wykorzystuje się głównie czasowniki, rzeczowniki, liczby i jednostki miar. Ogranicza się wykorzystanie przymiotników.
	\end{itemize}

\item Zestawienie w formie tabeli --- rozwiązanie jest prezentowane w formie tabeli/tabel, zwierających kluczowe dane potrzebne do rozwiązania zadania. Poza tabelą zawarty jest krótki opis celu.

Kryteria tworzenia:
	\begin{itemize}
		\item Kluczowe dane liczbowe są przedstawione w formie tabeli z oznaczeniem nagłówków i wierszy.
		\item Pod tabelą znajdują się dodatkowe sekcje przedstawiające zasoby, które potrzebują dokładniejszego opisu, ograniczenia oraz cel optymalizacyjny w formie wypunktowanej.
		\item W opisie poniżej tabeli wykorzystuje się rzeczowniki, liczby oraz jednostki miar.
	\end{itemize}
\item Opis z strukturalnym uporządkowaniem --- elementy w opisie są rozdzielane na kategorie, tworząc łatwe przejrzyste i łatwe do analizy segmenty.

Kryteria tworzenia:
	\begin{itemize}
		\item Wszystkie dane i elementy są podzielone na kategorie. Ostatnią kategorią jest cel optymalizacyjny.
		\item Każda kategoria posiada należące do niej elementy zapisane w formie podpunktów.
		\item Średnia długość podpunktu wynosi 4 słowa.
		\item W opisie wykorzystywane są rzeczowniki, liczby oraz jednostki miar.
	\end{itemize}

\end{enumerate}
\item Weryfikacja poprawności zadania - zadanie musi posiadać pełny opis, na podstawie którego można stworzyć docelowy model w \akronim{ZIMPL}. Usuwane są zadania, w których niespełniane są wymagania pozafunkcjonalne.
\item Proces zostaje powtórzony dla kolejnych zadań.
\end{enumerate}

Decyzję o takim zwielokrotnianiu posiadanych podjęto z kilku powodów. Pierwszym z nich była potrzeba weryfikacji, czy stworzony model generatora kodu \akronim{ZIMPL} poradzi sobie z każdym rodzajem zadania, niezależnie od jego formuły i kolejności przekazywanych danych. Drugim kluczowym argumentem była mała dostępność do treści zagadnień optymalizacyjnych i modeli PL połączona z zapotrzebowaniem na dużą ilość danych do testów.

\section{Podział i wykorzystanie zbioru danych}

\begin{table}[H]
\caption{Podział zbioru danych na podzbiory.}\label{tab:dataset:stats}
\centering%
\begin{tabular}{|l|c|c|c|}
\hline
\textbf{Zbiór} & \textbf{Sztywne} & \textbf{Parametryzowane} & \textbf{Razem} \\
\hline
Treningowy & 10 & 10 & 20\\
\hline
Walidacyjny & 30 & 30 & 60\\
\hline
Testowy & 2275 & 2025 & 4300\\
\hline
\textbf{Łącznie} & \textbf{2315} & \textbf{2065} & \textbf{4380}\\
\hline
\end{tabular}
\end{table}

Łącznie stworzono około 4800 przykładów zagadnień optymalizacyjnych wraz z~modelami PL w~języku \akronim{ZIMPL}. Po ręcznej analizie zadań i automatycznym sprawdzeniu rozwiązań, zdecydowano się usunąć niekonkretne lub trudne do interpretacji przykłady, a także przykłady, dla których rozwiązanie dostarczone przez SCIP zostało oznaczone jako `niepoprawne'. Przy takiej selekcji zadań, zbiór pomniejszył się do 4380 przykładów. Tak utworzony zbiór danych został podzielony na następujące podzbiory: treningowy, walidacyjny oraz testowy. Z racji, że generator nie wymagał bezpośredniego uczenia automatycznego, a ręcznego strojenia zapytań do DMJ, liczebność poszczególnych zbiorów znacząco się różniła. Do trenowania potrzebne było 20 przykładów, tak aby nie zapełniać zapytań powtarzającymi się schematami odpowiedzi. Zbiór walidacyjny został dobrany z bazowych przykładów w taki sposób, aby uruchamianie generatora było nisko kosztowne i pozwalało na realizację walidacji w krótkim czasie (około 20 minut). Dla porównania, wygenerowanie odpowiedzi \akronim{ZIMPL} dla zbioru testowego trwało ponad 16 godzin. Dokładne statystyki podzbiorów podano w Tabeli~\ref{tab:dataset:stats}. 

\subsection{Zbiór treningowy}

Pierwszy zbiór nazwany treningowym, jest najmniejszym z utworzonych zbiorów i został wykorzystany bezpośrednio do tworzenia zapytań do dużego modelu językowego. Zbiór treningowy nie jest używany w walidacji i testach, bo model językowy posiada dokładne odpowiedzi do zawartych treści. Przykłady zostały dobrane w sposób różnorodny, tak aby jak najlepiej przedstawić DMJ oczekiwane rezultaty generacji kodu \akronim{ZIMPL}. Wybór tego zbioru odbył się w sposób ręcznego przeglądu 40 bazowych przykładów i wyboru po 10 najbardziej zdywersyfikowanych pod względem składni kodu \akronim{ZIMPL} modeli programowania sztywnego i parametryzowanego.

Zbiór treningowy jest mały, aby pomieścić zawartość wybranych przykładów w pojedynczym zapytaniu do DMJ, którego rozmiar jest ograniczony możliwościami DMJ.

\subsection{Zbiór walidacyjny}

Do ręcznej walidacji zapytań podczas prototypowania generatora kodu  \akronim{ZIMPL} wykorzystano początkowo 60 przykładów, w tym 30 przykładów programowania sztywnego i 30 przykładów programowania parametryzowanego. Wybrano pozostałe 30 opisów problemów PL ze zbioru bazowego, uszczuplonego o zbiór treningowy, wraz z modelami \akronim{ZIMPL} w dwóch wersjach. Przy pomocy tego zbioru sprawdzano jakość generowanego kodu  \akronim{ZIMPL} oraz weryfikowano jego poprawność pod względem posiadanego rozwiązania oraz pokrycia rozwiązania z rozwiązaniem zawartym w zbiorze.

Zdecydowano się na taką liczbę przykładów w zbiorze ze względu na potrzebę szybkiej weryfikacji wyników. Czas generowania wszystkich 60 odpowiedzi wynosił maksymalnie 7 minut, co sprawiło, że można było szybko sprawdzać i naprawiać błędy w logice zapytań. Zbiór celowo zawarł w sobie połowę zadań z wynikami parametryzowanymi oraz połowę zadań z wynikami sztywno zaprogramowanymi, tak aby zapewnić poprawny wgląd do rozwiązań obu problemów. Zadania zostały wyselekcjonowane w różnorodny sposób, tak aby dać możliwie najszerszy wgląd w problemy, z jakimi może się mierzyć generator.

\subsection{Zbiór testowy}

Ostatni i zarazem największy wydzielony zbiór jest zbiorem testowym. Znajduje się w nim pozostałe 4300 przykładów, w tym 2315 przykładów sztywnego i 2065 przykładów parametryzowanego kodu. Zbiór został wykorzystany do prowadzenia wyłącznie zautomatyzowanych testów. Na jego podstawie zostały opracowane wyniki eksperymentu w Rozdziale~\ref{ch:experiment}.

Zbiór został stworzony tak, aby liczba przykładów odpowiedniego sposobu programowania była porównywalna, natomiast w procesie selekcji i usuwania przykładów ta liczba nieznacznie się zmieniła.

\section{Przechowywanie i dostępność zbioru}

W związku ze specyficznym formatowaniem zbiory są przechowywane w formie pliku arkusza kalkulacyjnego, a następnie konwertowane do pliku JSON, z którego pobierane są do wykonywania walidacji i testów generatora. Zbiór treningowy znajduje się bezpośrednio w zapytaniach do modelu językowego. Pozostałe dwa zbiory w formacie JSON znajdują się również na platformie Hugging Face\footnote{\label{fn:dataset:link}\url{https://huggingface.co/datasets/Tamiza/test_zimpl_dataset}} na licencji MIT, gdzie można je darmowo wykorzystywać.