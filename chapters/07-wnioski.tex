
\chapter{Wnioski i dalszy kierunek prac}\label{ch:conclusions}

\section{Wnioski}
%Na początku pracy postawiono hipotezę badawczą:
%\begin{quote}
%\textit{Można wykorzystać duże modele językowe do generowania modeli PL zapisanych w języku naturalnym.}
%\end{quote}

W ramach pracy przygotowano unikalny zbiór danych zawierający 4380 przykładów opisów zagadnień optymalizacyjnych oraz związanych z~nimi wzorcowych modeli PL w języku ZIMPL i~udostępniono go publicznie.$^\text{\ref{fn:dataset:link}}$ Drugim przyczynkiem jest generator modeli Programowania Liniowego (PL) z opisów zagadnień optymalizacyjnych w~języku angielskim bazujący na Dużym Modelu Językowym (DMJ). % TP: świadomie powtarzam definicję skrótów PL i DMJ, bo czytelnik może czasem przeczytać wstęp i koniec i nie będzie się wgłębiać w szczegóły w środku
Wyniki przeprowadzonych eksperymentów w~Rozdziale~\ref{ch:experiment} potwierdzają empirycznie główną hipotezę badawczą z Rozdziału~\ref{sec:intro:aim}: generator bazujący na DMJ tworzy poprawne modele PL z opisów tekstowych z~prawdopodobieństwem ponad 60\%. % TP: mam na myśli tabelę 5.4 - bo jednak chodzi tutaj o globalny wskaźnik sukcesu - 60 prób na 100 powiedzie się.

%Analizując przedstawione rozwiązanie i wspomagając się eksperymentami założono poprawność powyższej hipotezy. Przy odpowiednio przygotowanym zapytaniu można z dużą pewnością założyć, że duże modele językowe można wykorzystać do generowania modeli PL otrzymując na wejściu jedynie problem zapisany w języku naturalnym. 
Uzyskane wyniki są zależne od informacji kontekstowej dostarczonej do DMJ w zapytaniu. Zagadnienie inżynierii zapytań nie jest trywialne i możliwe, że istnieją inne, bardziej skuteczne zapytania od zaproponowanych, z~dużą pewnością zależne od użytego DMJ. 
Dla opracowanych zapytań, istnieje znacząca poprawa w~jakości wygenerowanego modelu PL w~stosunku do modeli PL generowanych za pomocą pojedynczego prostego zapytania. Zaobserwowano wzrost skuteczności DMJ przy generowaniu modeli \textit{PL}. Przekłada się to na usprawnienie procesu modelowania optymalizacyjnego. 

Wyniki eksperymentu pokazały, iż bardziej precyzyjne oraz szczegółowo sformułowane zapytania, przekładają się bezpośrednio na wzrost jakości wygenerowanego modelu \textit{PL} przez \textit{DMJ}. Redukuje to konieczność kosztownej w zasobach i czasie pracy ekspertów. Rozwiązanie to zmniejsza liczbę błędów oraz redukuje niedoskonałości wykonania wynikające z czynnika ludzkiego. 

Zauważono również, iż zwiększenie różnorodności przykładów w zapytaniach przyczyniłoby się do poprawy jakości generowanych modeli \textit{PL} przez DMJ. Tworzy to dodatkowe adaptacje dla DMJ oraz polepsza ich elastyczność. % TP: TODO: adaptacje?

\section{Dalszy kierunek prac}

Kolejnym etapem rozwoju sytemu generującego modele \textit{PL} byłoby wzbogacenie zbioru danych o~bardziej różnorodne przykłady. Dodanie większej liczby niestandardowych zagadnień optymalizacyjnych może istotnie poszerzyć klasę generowany poprawnie modeli PL.

Istotnym rozważenia jest również adaptacja alternatywnych narzędzi typu \textit{solver} takich jak \textit{Gurobi} \cite{TODO}. Poszerzyłoby to horyzonty w implementacji użytego rozwiązania na nowe obszary dziedziny optymalizacji. Istnieje również możliwość opracowania narzędzi do wizualizacji, pozwalających na graficzną reprezentację problemów \textit{PL}, co ułatwiłoby odbiór oraz interpretacje wygenerowanego rozwiązania. 

Skoncentrowanie się na poszerzeniu zakresu zbioru danych, również może zwiększyć dokładność generowanych modeli \textit{PL} przez \textit{DMJ}. 