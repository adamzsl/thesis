
\chapter{Wnioski i dalszy kierunek prac}

\section{Wnioski}
Na początku pracy postawiono hipotezę badawczą:
\begin{quote}
\textit{Można wykorzystać duże modele językowe do generowania modeli PL zapisanych w języku naturalnym.}
\end{quote}

Analizując przedstawione rozwiązanie i wspomagając się eksperymentami założono poprawność powyższej hipotezy. Przy odpowiednio przygotowanym zapytaniu można z dużą pewnością założyć, że duże modele językowe można wykorzystać do generowania modeli PL otrzymując na wejściu jedynie problem zapisany w języku naturalnym. Istnieje znacząca poprawa w jakości wygenerowanego rozwiązania w stosunku do rozwiązania za pomocą pojedynczego, prostego zapytania. Zaobserwowano wzrost skuteczności dużych modeli językowych przy generowaniu modeli \textit{PL}. Przekłada się to na usprawnienie procesu modelowania optymalizacyjnego. 

Wyniki eksperymentu pokazały, iż bardziej precyzyjne oraz szczegółowo sformułowane zapytania, przekładają się bezpośrednio na wzrost jakości wygenerowanego modelu \textit{PL} przez \textit{DMJ}. Redukuje to konieczność kosztownej w zasobach i czasie pracy ekspertów. Rozwiązanie to zmniejsza liczbę błędów oraz redukuje niedoskonałości wykonania wynikające z czynnika ludzkiego. 

Zauważono również, iż zwiększenie różnorodności przykładów przyczyniłoby się do poprawy jakości generowanych modeli \textit{PL} przez duże modele językowe. Tworzy to dodatkowe adaptacje dla dużych modeli językowych oraz polepsza ich elastyczność.

\section{Dalszy kierunek prac}

Kolejnym etapem rozwoju sytemu generującego modele \textit{PL} byłoby wzbogacenie bazy danych o bardziej różnorodne przykłady. Dodanie większej ilości niestandardowych problemów może istotnie poprawić jakość generowanych rozwiązań.

Istotnym rozważenia jest również adaptacja alternatywnych narzędzi typu \textit{solver} takich jak \textit{Gurobi}. Poszerzyłoby to horyzonty w implementacji użytego rozwiązania na nowe obszary dziedziny optymalizacji. Istnieje również możliwość opracowania narzędzi do wizualizacji, pozwalających na graficzną reprezentację problemów \textit{PL}, co ułatwiłoby odbiór oraz interpretacje wygenerowanego rozwiązania. 

Skoncentrowanie się na poszerzeniu zakresu zbioru danych, również może zwiększyć dokładność generowanych modeli \textit{PL} przez \textit{DMJ}. 