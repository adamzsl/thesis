
\chapter{Wprowadzenie}\label{ch:intro}

\section{Motywacja}\label{sec:intro:motivation}

Programowanie Liniowe \english{Linear Programmimng, PL} jest popularną metodą modelowania zagadnień optymalizacyjnych, szczególnie w badaniach operacyjnych \cite{williams2013model}. PL ma liczne zastosowania, np.~w logistyce, planowaniu łancuchów dostaw oraz optymalizacji zużycia materiałów~\cite{dantzig2002linear}.

Model PL składa się ze zmiennych odpowiadających zmiennym czynnikom modelowanego zagadnienia i~specyfikowanych wraz z~dziedzinami (ciągłe, całkowitoliczbowe, z określonego przedziału); ograniczeń wyrażających cechy zagadnienia w~kategoriach zależności liniowych pomiędzy zmiennymi; funkcji celu określającej pożądane cechy rozwiązania jako funkcję liniową zmiennych. Na potrzeby tej pracy, definicję modelu PL rozszerzono o~zbiory wartości odpowiadające wymiarom modelowanego zagadnienia oraz parametry odpowiadające stałym wartościom liczbowym. Są to czynniki umożliwiające wysokopoziomowy zapis modelu w~abstrakcji od wartości liczbowych występujących w konkretnej instancji zagadnienia. Wówczas implementację modelu PL dla konkretnej instancji realizuje się poprzez podstawienie wartości pod zbiory i~parametry.
Wektor wartości zmiennych nazywany jest \emph{rozwiązaniem} modelu PL. Rozwiązanie jest \emph{dopuszczalne} jeśli spełnia wszystkie ograniczenia. Rozwiązanie dopuszczalne jest \emph{optymalne}, jeśli minimalizuje lub maksymalizuje funkcję celu.

Do rozwiązywania modeli PL istnieje wiele solverów, zarówno komercyjnych takich jak Gurobi~\cite{gurobi2023} czy CPLEX~\cite{cplex2019}, jak i otwartoźródłowych, takich jak GLPK~\cite{glpk2023}, COIN-OR CBC~\cite{cbc2023} oraz SCIP~\cite{scip2023}. 

Korzystanie z solwerów jest do pewnego stopnia ustandaryzowane dzięki wsparciu dla wysokopoziomowych języków zapisu modeli PL, np.:~\akronim{AMPL} \cite{fourer2003ampl}, \akronim{GAMS} \cite{gams2019}, \akronim{ZIMPL} \cite{zimpl_manual}, \akronim{MiniZINC} \cite{nethercote2007minizinc}. Wymienione języki obsługują wszystkie wyżej wymienione elementy modelu PL, a~ich interpretery transformują model PL zapisany w danym języku do postaci rozumianej przez solwer, np.~wywołań z API, albo formatów LP i MPS.

Największa trudnością jest sporządzenie ów modeli PL --- aktualnie buduje się je głównie ręcznie, często na podstawie opisu tekstowego zagadnienia optymalizacyjnego wyrażonego w języku naturalnym.  Pomimo, że język naturalny często nie jest tak precyzyjny jak wyrażenia matematyczne stosowane w~modelu PL, eksperci modelowania są w~stanie utworzyć na jego podstawie modele PL o~zadowalającej jakości.
Ręczne modelowanie jest czasochłonne, a~zmiany w~zagadnieniu optymalizacyjnym wymagają często nietrywialnych modyfikacji modelu PL oraz są potencjalnym źródłem błędów czynnika ludzkiego. Modelowanie wymaga wiedzy na temat składni określonego języka modelowania, sposobów liniowego modelowania nieliniowych zależności, oraz wiedzy dziedzinowej nt.~zagadnienia.
Ponadto, modele PL muszą być poprawne semantycznie. 

Dla złożonych zagadnień definiowanych prez duże zbiory danych i skomplikowane zależności, błędy w~semantyce modeli bardzo trudno zidentyfikować. Wymaga to ponownej analizy kodu i~nierzadko sporządzenia modelu od nowa.

Opis zagadnienia optymalizacyjnego językiem naturalnym jest przystępną formą stawiania wymagań, nawet dla laików optymalizacji. Dostępność takiego opisu otwiera możliwość użycia Dużego Modelu Językowego (\akronim{DMJ}) \english{Large Language Model, LLM} \cite{brown2020language} jako narzędzia generowania modeli PL wspomagającego ekspertów optymalizacji.

\section{Cel i zakres pracy}\label{sec:intro:aim}

W pracy podjęto próbę weryfikacji następującej hipotezy badawczej:
\begin{quote}
\textit{Duży model językowy może wygenerować model PL w wysokopioziomowym języku modelowania na podstawie opisu zagadnienia optymalizacyjnego w języku naturalnym.}
\end{quote}

Hipoteza zostanie zweryfikowana poprzez próbę opracowania systemu korzystającego z \textit{DMJ} do automatycznej generacji modeli \textit{PL} na podstawie słownych opisów problemów optymalizacyjnych w~języku angielskim. System analizując dostarczone opisy, będzie generował kod w języku \textit{ZIMPL} identyfikując zbiory danych, parametry, zmienne decyzyjne, funkcje celu oraz ograniczenia. Kody będą generowane na dwa sposoby: tzw.~\textit{hardcoded} oraz parametryzowane. Docelowo narzędzie będzie wspomagało proces generowania modeli problemów optymalizacyjnych, przyczyniając się do usprawnienia pracy specjalistów w dziedzinie optymalizacji, pozwalając na szybsze oraz mniej podatne na błędy generowanie kodu.

Jako reprezentację modelu PL wybrano język \textit{ZIMPL} \cite{zimpl_manual}, ponieważ jego interpreter jest otwartoźródłowy, a~język jest elastyczny i~szeroko stosowany w badaniach operacyjnych. Dodatkowo, analiza popularnych narzędzi \textit{DMJ} wykazała, że generowany kod w tym języku czasami zawiera błędy składniowe, co otwiera pole do prac badawczych nad poprawą działania DMJ dla tego jezyka.

Na szczegółowy zakres prac składają się:
\begin{itemize}
    \item Przegląd literatury dotyczącej \textit{PL}.
    \item Budowa bazy danych opisów zagadnień optymalizacyjnych i wzorcowych modeli PL w języku \textit{ZIMPL}.
    \item Budowa systemu korzystającego z \textit{DMJ} w celu generacji kodu na podstawie dostarczonych instrukcji.
    \item Budowa systemu wykorzystującego solver oraz \textit{DMJ} do oceny kodu \textit{PL}.
\end{itemize}

\section{Podział pracy}

Jagoda Janowska była odpowiedzialna za zwielokrotnianie przykładów w zbiorze danych pod względem modyfikacji kodu i sformułowań zadania. Na podstawie zbioru testowego stworzyła zapytania przeznaczone do generacji poszczególnych elementów modelu \textit{ZIMPL}. Odpowiadała za stworzenie przepływu zapytań, a także narzędzia do generacji i testów w środowisku \textit{Jupyter Notebook}. Stworzyła graficzny interfejs użytkownika. Sporządziła elementy części teoretycznej oraz część eksperymentalną pracy dyplomowej.

Adam Maciejak był zaangażowany w wyszukiwanie oryginalnych opisów problemów \textit{PL} oraz ręczne tworzenie modeli w języku \textit{ZIMPL}. Tworzył on system oceny wyjściowych modeli \textit{PL} używający \textit{DMJ} oraz przygotował narzędzia do ich walidacji. Sporządził teoretyczną część niniejszej pracy dyplomowej.

Adam Mikołajczak wyszukiwał oryginalne opisy problemów \textit{PL} oraz tworzył modele w języku \textit{ZIMPL}. Tworzył system oceny wyjściowych modeli \textit{PL} korzystający z \textit{DMJ}. Utworzył programy walidujące oraz rozwiązujące kod w języku \textit{ZIMPL} korzystając z narzędzia \textit{SCIP}. Usprawnił proces walidacji rozwiązań w języku \textit{ZIMPL}, łącząc się poprzez API z Google Sheets. Sporządził teoretyczną część niniejszej pracy dyplomowej. 

\section{Struktura Pracy}

Struktura pracy jest następująca. W~rozdziale~\ref{ch:review} przedstawiono przegląd literatury na temat programowania liniowego i systemów generujących modele PL z~tekstu. Rozdział~\ref{ch:generation} jest poświęcony opisowi opracowanego systemu. Rozdział~\ref{ch:dataset} zawiera opis zbioru danych. Tłumaczy on strukturę pracy, tworzenie odpowiedzi w formacie kodu \akronim{ZIMPL} oraz opisuje przebieg rozwoju oraz przetwarzania zbioru danych. W~rozdziale~\ref{ch:experiment} przedstawiono pytania eksperymentalne oraz odpowiedzi na nie, wraz ze statystykami przebiegu eksperymentów. Rozdział~\ref{ch:discussion} omawia zalety oraz wady rozwiązania, proces poprawy rozwiązania jak i napotkane trudności związane z~przebiegiem pracy. W~rozdziale~\ref{ch:conclusions} znajduje się podsumowanie zawierające wnioski oraz dalszy kierunek prac.
