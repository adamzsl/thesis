
\chapter{Wprowadzenie}

\section{Motywacja (1-1,5 strony)}

Programowanie liniowe (\textit{PL}) jest metodą minimalizacji wieloaspektowej z funkcją kosztu. Metodę opisują poniższe twierdzenia\footnote{Definicja i twierdzenia pochodzą częściowo z Metod Podejmowania Decyzji Waldemara Rebizanta~\cite{rebizant2012metody}.}:
\begin{itemize}
\item Tw1: Zbiór wypukły \textit{W} to zbiór, w którym odcinek łączący dwa dowolne punkty zawierające się w zbiorze \textit{W}, również zawiera się w zbiorze \textit{W}.

\item Tw2: Zbiór rozwiązań dopuszczalnych \textit{D} jest zbiorem wypukłym.

\item Tw3: Warunki ograniczające tworzą wierzchołek zbioru wypukłego \textit{D}, gdzie funkcja osiąga wartość optymalną.

\item Tw4: Przy co najmniej dwóch rozwiązaniach optymalnych, każda liniowa kombinacja tych rozwiązań, jest rozwiązaniem optymalnym danego modelu decyzyjnego.
\end{itemize}
Programowanie liniowe jest jedną z najbardziej popularnych metod badań optymalizacyjnych. Ma ono zastosowanie w szeroko pojętej logistyce np. planowanie łancuchów dostaw czy optymalizacja zużycia materiałów.\footnote{Informacje pochodzą z artykułu opublikowanego przez twórcę programowania liniowego George'a B. Dantziga\cite{dantzig2002linear}} Do rozwiązywania problemów programowania liniowego istnieje wiele solverów, zarówno komercyjnych jak i rozwiązań do użytku publicznego. Solvery rozwiązują modele programowania liniowego, które mogą być zapisane za pomocą różnych języków programowania z ograniczeniami. Do popularnych języków należą: \textit{AMPL}, \textit{GAMS}, \textit{ZIMPL}, \textit{MiniZINC}.

Największa trudnością jest sporządzenie ów modeli --- aktualnie buduje się je głównie ręcznie. Jest to bardzo czasochłonne, każda zmiana w procesie logistycznym wymaga ponownego przyjrzenia się problemowi oraz jest wrażliwa na błędy wynikające z czynnika ludzkiego. Wymaga to precyzyjnej wiedzy na temat składni określonego języka modelowania.

Dodatkowo modele muszą być poprawne logicznie. Przy złożonych problemach operujących na bardzo dużych danych i skomplikowanych ograniczeniach, błędy w logice modeli bardzo trudno zidentyfikować. Wymaga to ponownej analizy kodu, nierzadko sporządzenia modelu od nowa.

Rozważono użycie Dużego Modelu Językowego (\textit{DMJ}) do generowania modeli programowania liniowego w języku \textit{ZIMPL} na podstawie opisów sporządzonych w języku naturalnym.

\textit{DMJ} wygeneruje kod, który będzie składał się z 5 głównych sekcji: \textit{zbiór danych}, \textit{parametrów}, \textit{zmiennych decyzyjnych}, \textit{funkcji celu} oraz \textit{ograniczeń}. Wyróżniliśmy dwa formaty kodu wynikowego: kod parametryzowany oraz kod tzw. \textit{hardcoded}. 

Wybrano język \textit{ZIMPL}, ponieważ w przeciwności do wcześniej wymienionych jest on \textit{open source} oraz \textit{DMJ} błędnie generują kod w tym języku. Są to w większości błędy w składni.

\section{Cel i zakres pracy (0,5 strony)}

\subsection{Cel}
Celem pracy jest opracowanie systemu korzystającego z \textit{DMJ} do automatycznej generacji modeli \textit{PL} na podstawie słownych opisów problemów optymalizacyjnych. System analizując dostarczone opisy, będzie generował kod w języku \textit{ZIMPL} identyfikując zbiory danych, parametry, zmienne decyzyjne, funkcje celu oraz ograniczenia. Kody będą generowane na dwa sposoby: tzw. \textit{hardcoded} oraz parametryzowane. Docelowo narzędzie będzie wspomagało proces generowania modeli problemów optymalizacyjnych, przyczyniając się do usprawnienia pracy specjalistów w dziedzinie optymalizacji, pozwalając na szybsze oraz mniej podatne na błedy generowanie kodu.

\subsection{Zakres pracy}
\begin{itemize}
    \item Przegląd literatury dotyczącej \textit{PL}.
    \item Budowa bazy danych opisów problemów optymalizacyjnych i wzorcowych rozwiązań kodu w języku \textit{ZIMPL}.
    \item Budowa systemu korzystającego z \textit{DMJ} w celu generacji kodu na podstawie dostarczonych instrukcji.
    \item Budowa systemu wykorzystującego solver oraz \textit{DMJ} do oceny kodu \textit{PL}.
\end{itemize}

\subsection{Hipoteza badawcza} 
\begin{quote}
\textit{Można wykorzystać duże modele językowe do generowania modeli PL zapisanych w języku naturalnym.}
\end{quote}

\subsection{Struktura Pracy}
TODO

Struktura pracy jest następująca. W rozdziale 2 przedstawiono przegląd literatury na temat ........ 
Rozdział 3 jest poświęcony ....... (kilka zdań). 
Rozdział 4 zawiera ..... (kilka zdań) ............ itd. 
Rozdział X stanowi podsumowanie pracy. 


\section{Podział pracy (0,5 strony)}
TODO

Jan Kowalski w ramach niniejszej pracy wykonał projekt tego i tego, opracował ......

Grzegorz Brzęczyszczykiewicz wykonał ......, itd. 

