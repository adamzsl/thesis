
\chapter{Wprowadzenie}

\section{Motywacja}
Motywacja (co to jest programowanie liniowe, do czego się przydaje, jego powszechność, zastosowanie), cytowanie – scholar google

Problemy programowania liniowego (są solvery, ale problem: budowanie modelu – aktualnie buduje się ręcznie, czasochłonne zadanie, wiele iteracji). Popularny język AMPL; GAMS; ZIMPL; MiniZINC (programowanie z ograniczeniami).

Podatność na błędy w modelach – trudno zidentyfikować

Propozycja: AI generujące modele na podstawie tekstowych opisów

Przykład i oczekiwania w temacie poszczególnych elementów kodu (ograniczenia, etc). Trudności, które pojawiły się w trakcie 
    realizacji poszczególnych zadań, uwagi dotyczące wykorzystywanego sprzętu

Krótkie uzasadnienie podjęcia tematu

\section{Cel i zakres pracy}
Cel pracy. Konkretnie co robimy, opracowanie system korzystający z LLM. Zakres (przedmiotowy, podmiotowy, czasowy) wyjaśniający, w jakim rozmiarze praca będzie realizowana

Hipoteza badawcza: Można wykorzystać duże modele językowe do generowania modeli PL zapisanych w języku naturalnym.
\begin{itemize}
    \item budowa bazy danych opisów problemów optymalizacyjnych i wzorcowych rozwiązań
    \item budowa systemu wykorzystującego LLM do PL 
    \item budowa systemu wykorzystującego solver i LLM do oceny PL
\end{itemize}

\noindent
\textbf{Wstęp do pracy musi się kończyć dwoma następującymi akapitami:}

Celem pracy jest opracowanie / wykonanie analizy / zaprojektowanie / ...........

Struktura pracy jest następująca. W rozdziale 2 przedstawiono przegląd literatury na temat ........ 
Rozdział 3 jest poświęcony ....... (kilka zdań). 
Rozdział 4 zawiera ..... (kilka zdań) ............ itd. 
Rozdział X stanowi podsumowanie pracy. 


\section{Podział pracy}
Jan Kowalski w ramach niniejszej pracy wykonał projekt tego i tego, opracował ......

Grzegorz Brzęczyszczykiewicz wykonał ......, itd. 

