
\chapter{Wprowadzenie}

\section{Motywacja}

Programowanie liniowe jest jedną z najbardziej popularnych metod badań optymalizacyjnych. Ma ono zastosowanie w szeroko pojętej logistyce np. planowanie łancuchów dostaw czy optymalizacja zużycia materiałów.\footnote{Informacje pochodzą z artykułu opublikowanego przez twórcę programowania liniowego George'a B. Dantziga\cite{dantzig2002linear}} Do rozwiązywania problemów programowania liniowego istnieje wiele solverów, zarówno komercyjnych jak i rozwiązań do użytku publicznego. Solvery rozwiązują modele programowania liniowego, które mogą być zapisane za pomocą różnych języków programowania z ograniczeniami. Do popularnych języków należą: \textit{AMPL}, \textit{GAMS}, \textit{ZIMPL}, \textit{MiniZINC}.

Największa trudnością jest sporządzenie ów modeli --- aktualnie buduje się je głównie ręcznie. Jest to bardzo czasochłonne, każda zmiana w procesie logistycznym wymaga ponownego przyjrzenia się problemowi oraz jest wrażliwa na błędy wynikające z czynnika ludzkiego. Wymaga to precyzyjnej wiedzy na temat składni określonego języka modelowania.

Dodatkowo modele muszą być poprawne logicznie. Przy złożonych problemach operujących na bardzo dużych danych i skomplikowanych ograniczeniach, błędy w logice modeli bardzo trudno zidentyfikować. Wymaga to ponownej analizy kodu, nierzadko sporządzenia modelu od nowa.

Rozważono użycie Dużego Modelu Językowego (\textit{DMJ}) do generowania modeli programowania liniowego w języku \textit{ZIMPL} na podstawie opisów sporządzonych w języku naturalnym.

\textit{DMJ} wygeneruje kod, który będzie składał się z 5 głównych sekcji: \textit{zbiór danych}, \textit{parametrów}, \textit{zmiennych decyzyjnych}, \textit{funkcji celu} oraz \textit{ograniczeń}. Wyróżniliśmy dwa formaty kodu wynikowego: kod parametryzowany oraz kod tzw. \textit{hardcoded}. 

Wybrano język \textit{ZIMPL}, ponieważ w przeciwności do wcześniej wymienionych jest on \textit{open source} oraz \textit{DMJ} błędnie generują kod w tym języku. Są to w większości błędy w składni.

\section{Cel i zakres pracy}

\subsection{Cel}
Celem pracy jest opracowanie systemu korzystającego z \textit{DMJ} do automatycznej generacji modeli \textit{PL} na podstawie słownych opisów problemów optymalizacyjnych. System analizując dostarczone opisy, będzie generował kod w języku \textit{ZIMPL} identyfikując zbiory danych, parametry, zmienne decyzyjne, funkcje celu oraz ograniczenia. Kody będą generowane na dwa sposoby: tzw. \textit{hardcoded} oraz parametryzowane. Docelowo narzędzie będzie wspomagało proces generowania modeli problemów optymalizacyjnych, przyczyniając się do usprawnienia pracy specjalistów w dziedzinie optymalizacji, pozwalając na szybsze oraz mniej podatne na błedy generowanie kodu.

\subsection{Zakres pracy}
\begin{itemize}
    \item Przegląd literatury dotyczącej \textit{PL}.
    \item Budowa bazy danych opisów problemów optymalizacyjnych i wzorcowych rozwiązań kodu w języku \textit{ZIMPL}.
    \item Budowa systemu korzystającego z \textit{DMJ} w celu generacji kodu na podstawie dostarczonych instrukcji.
    \item Budowa systemu wykorzystującego solver oraz \textit{DMJ} do oceny kodu \textit{PL}.
\end{itemize}

\subsection{Hipoteza badawcza} 
\begin{quote}
\textit{Można wykorzystać duże modele językowe do generowania modeli PL zapisanych w języku naturalnym.}
\end{quote}

\section{Podział pracy}

Jagoda Janowska była odpowiedzialna za zwielokrotnianie przykładów w zbiorze danych pod względem modyfikacji kodu i sformułowań zadania. Na podstawie zbioru testowego stworzyła zapytania przeznaczone do generacji poszczególnych elementów modelu \textit{ZIMPL}. Odpowiadała za stworzenie przepływu zapytań, a także narzędzia do generacji i testów w środowisku \textit{Jupyter Notebook}. Stworzyła graficzny interfejs użytkownika. Sporządziła elementy części teoretycznej oraz część eksperymentalną pracy dyplomowej.

Adam Maciejak był zaangażowany w wyszukiwanie oryginalnych opisów problemów \textit{PL} oraz ręczne tworzenie modeli w języku \textit{ZIMPL}. Tworzył on system oceny wyjściowych modeli \textit{PL} używający \textit{DMJ} oraz przygotował narzędzia do ich walidacji. Sporządził teoretyczną część niniejszej pracy dyplomowej.

Adam Mikołajczak wyszukiwał oryginalne opisy problemów \textit{PL} oraz tworzył modele w języku \textit{ZIMPL}. Tworzył system oceny wyjściowych modeli \textit{PL} korzystający z \textit{DMJ}. Utworzył programy walidujące oraz rozwiązujące kod w języku \textit{ZIMPL} korzystając z narzędzia \textit{SCIP}. Usprawnił proces walidacji rozwiązań w języku \textit{ZIMPL}, łącząc się poprzez API z Google Sheets. Sporządził teoretyczną część niniejszej pracy dyplomowej. 
